%摘要
\section*{动态规划中的各种背包问题}
本篇模板主要阐述动态规划中的背包问题,以及其解法思路。


\section{问题说明}

\subsection{问题系列与共性}
背包问题是动态规划中的经典问题,属于组合优化问题。它具有最优子结构性质和重叠子问题特征,是理解动态规划思想的重要基础。该问题广泛应用于资源分配、投资决策、任务调度等实际场景。

\subsection{核心问题描述}
给定一个容量为W的背包和n个物品,每个物品有重量$w_i$和价值$v_i$。每个物品只能选择一次(要么拿,要么不拿),求在不超过背包容量的前提下,能够获得的最大价值。

形式化描述:
\begin{align}
\max \sum_{i=1}^{n} v_i x_i \\
\text{s.t.} \sum_{i=1}^{n} w_i x_i \leq W \\
x_i \in \{0, 1\}
\end{align}

\subsection{问题本质与动态规划建模}
背包问题的本质是在约束条件下的最优化选择。通过将问题分解为子问题,我们可以用动态规划来解决:对于每个物品,我们面临"选择"或"不选择"的决策,而这个决策的最优性依赖于之前决策的最优解。

\section{0-1背包的动态规划解法}

\subsection{状态定义与转移约束}
定义状态:$dp[i][w]$表示前i个物品在背包容量为w时能获得的最大价值。

状态转移约束:
\begin{itemize}
\item 边界条件:$dp[0][w] = 0$(没有物品时价值为0)
\item 容量约束:$w \geq 0$且$w \leq W$
\item 物品约束:每个物品只能选择一次
\end{itemize}

\subsection{状态转移方程}
\begin{align}
dp[i][w] = \begin{cases}
dp[i-1][w] & \text{if } w_i > w \\
\max(dp[i-1][w], dp[i-1][w-w_i] + v_i) & \text{if } w_i \leq w
\end{cases}
\end{align}

伪代码:
\begin{verbatim}
Algorithm: 0-1Knapsack(weights[], values[], W, n)
    // 初始化DP表
    for i = 0 to n:
        for w = 0 to W:
            dp[i][w] = 0
    
    // 填充DP表
    for i = 1 to n:
        for w = 1 to W:
            if weights[i-1] <= w:
                dp[i][w] = max(dp[i-1][w], 
                              dp[i-1][w-weights[i-1]] + values[i-1])
            else:
                dp[i][w] = dp[i-1][w]
    
    return dp[n][W]
\end{verbatim}

\subsection{实例分析}
考虑以下例子:
\begin{itemize}
\item 背包容量:W = 4
\item 物品:[(重量2,价值1),(重量1,价值2),(重量3,价值3),(重量2,价值3)]
\end{itemize}

通过DP表的计算过程,最终得到最大价值为5(选择物品2和物品4)。

\subsection{空间复杂度优化}
由于$dp[i][w]$只依赖于$dp[i-1][w]$,可以使用一维数组优化空间复杂度从$O(nW)$降到$O(W)$:

\begin{verbatim}
Algorithm: OptimizedKnapsack(weights[], values[], W, n)
    dp[0...W] = 0
    
    for i = 1 to n:
        for w = W down to weights[i-1]:
            dp[w] = max(dp[w], dp[w-weights[i-1]] + values[i-1])
    
    return dp[W]
\end{verbatim}

\section{算法推广}

\subsection{完全背包:物品无限供应场景}
\subsubsection{问题描述}
与0-1背包不同,完全背包中每个物品可以选择多次,直到背包装不下为止。

\subsubsection{LeetCode案例:零钱兑换II (LeetCode 518)}
\textbf{题目描述:}
给你一个整数数组coins表示不同面额的硬币,以及一个整数amount表示总金额。
请你计算并返回可以凑成总金额的硬币组合数。题目数据保证答案符合32位带符号整数。

\textbf{示例:}
\begin{verbatim}
输入:amount = 5, coins = [1, 2, 5]
输出:4
解释:有四种方式可以凑成总金额:
5=5
5=2+2+1
5=2+1+1+1
5=1+1+1+1+1
\end{verbatim}

\textbf{解法思路:}
这是一个典型的完全背包问题。每种硬币可以使用无限次,目标是求组合数而非最大价值。
定义$dp[i]$表示凑成金额i的组合数。

\textbf{状态转移方程:}
\begin{align}
dp[i] = \sum_{\text{coin} \in \text{coins}, \text{coin} \leq i} dp[i - \text{coin}]
\end{align}

\textbf{伪代码实现:}
\begin{verbatim}
Algorithm: CoinChangeII(coins[], amount)
    dp[0...amount] = 0
    dp[0] = 1  // 凑成0元有1种方法
    
    for each coin in coins:
        for i = coin to amount:
            dp[i] += dp[i - coin]
    
    return dp[amount]
\end{verbatim}

\subsection{多重背包:物品有限数量场景}
\subsubsection{问题描述}
每个物品有特定的数量限制$num_i$,最多只能选择$num_i$个该物品。

\subsubsection{LeetCode案例:单词拆分 (LeetCode 139)}
\textbf{题目描述:}
给你一个字符串s和一个字符串列表wordDict,判断是否可以利用字典中出现的单词拼接出s。
注意:不要求字典中出现的单词全部都使用,并且字典中的单词可以重复使用。

\textbf{示例:}
\begin{verbatim}
输入: s = "leetcode", wordDict = ["leet","code"]
输出: true
解释: 返回 true 因为 "leetcode" 可以由 "leet" 和 "code" 拼接成。
\end{verbatim}

\textbf{解法思路:}
虽然看起来不像背包问题,但本质上是:字符串s是"背包",字典中的单词是"物品",每个单词可以重复使用。
定义$dp[i]$表示字符串前i个字符是否可以被拆分。

\textbf{状态转移方程:}
\begin{align}
dp[i] = \text{true if } \exists j < i \text{ such that } dp[j] = \text{true and } s[j:i] \in \text{wordDict}
\end{align}

\textbf{伪代码实现:}
\begin{verbatim}
Algorithm: WordBreak(s, wordDict)
    n = length(s)
    dp[0...n] = false
    dp[0] = true
    wordSet = convertToSet(wordDict)
    
    for i = 1 to n:
        for j = 0 to i-1:
            if dp[j] == true and s[j:i] in wordSet:
                dp[i] = true
                break
    
    return dp[n]
\end{verbatim}

\subsection{0-1背包变种:分割等和子集}
\subsubsection{LeetCode案例:分割等和子集 (LeetCode 416)}
\textbf{题目描述:}
给你一个只包含正整数的非空数组nums。请你判断是否可以将这个数组分割成两个子集,使得两个子集的元素和相等。

\textbf{示例:}
\begin{verbatim}
输入:nums = [1,5,11,5]
输出:true
解释:数组可以分割成 [1, 5, 5] 和 [11]。
\end{verbatim}

\textbf{解法思路:}
这是0-1背包问题的变种。如果数组能分割成两个等和子集,那么每个子集的和必须是总和的一半。
问题转化为:能否从数组中选择一些数字,使其和等于$\text{sum}/2$。

\textbf{状态转移方程:}
\begin{align}
dp[i][j] = dp[i-1][j] \text{ or } dp[i-1][j-\text{nums}[i-1]] \text{ (if } j \geq \text{nums}[i-1]\text{)}
\end{align}

\textbf{伪代码实现:}
\begin{verbatim}
Algorithm: CanPartition(nums[])
    sum = calculateSum(nums)
    if sum % 2 != 0:
        return false
    
    target = sum / 2
    dp[0...target] = false
    dp[0] = true
    
    for each num in nums:
        for j = target down to num:
            dp[j] = dp[j] OR dp[j - num]
    
    return dp[target]
\end{verbatim}

\subsection{扩展:目标和问题}
\subsubsection{LeetCode案例:目标和 (LeetCode 494)}
\textbf{题目描述:}
给你一个整数数组nums和一个整数target。向数组中的每个整数前添加'+'或'-',然后串联起所有整数,可以构造一个表达式。
返回可以通过上述方法构造的、运算结果等于target的不同表达式的数目。

\textbf{示例:}
\begin{verbatim}
输入:nums = [1,1,1,1,1], target = 3
输出:5
解释:一共有 5 种方法让最终目标和为 3。
-1 + 1 + 1 + 1 + 1 = 3
+1 - 1 + 1 + 1 + 1 = 3
+1 + 1 - 1 + 1 + 1 = 3
+1 + 1 + 1 - 1 + 1 = 3
+1 + 1 + 1 + 1 - 1 = 3
\end{verbatim}

\textbf{解法思路:}
设正数和为P,负数和为N,则有:
\begin{align}
P - N &= \text{target} \\
P + N &= \text{sum} \\
\Rightarrow P &= \frac{\text{sum} + \text{target}}{2}
\end{align}

问题转化为:从数组中选择一些数字,使其和等于P的方案数。这是一个0-1背包计数问题。

\textbf{伪代码实现:}
\begin{verbatim}
Algorithm: FindTargetSumWays(nums[], target)
    sum = calculateSum(nums)
    if sum < abs(target) OR (sum + target) % 2 != 0:
        return 0
    
    bagSize = (sum + target) / 2
    dp[0...bagSize] = 0
    dp[0] = 1
    
    for each num in nums:
        for j = bagSize down to num:
            dp[j] += dp[j - num]
    
    return dp[bagSize]
\end{verbatim}

\section{写在最后}

\subsection{问题本质的统一}
所有背包问题的核心都是在约束条件下的最优化决策。无论是01背包、完全背包还是多重背包,都遵循动态规划的基本原理:最优子结构和状态转移。

\subsubsection{背包问题的通用解决思路}

在解决各类背包问题时,我们可以遵循以下通用思路:

\textbf{1. 问题识别与转化}
\begin{itemize}
\item \textbf{约束识别:}明确"容量"限制(可能是重量、时间、空间等)
\item \textbf{选择识别:}确定"物品"概念(可能是实际物品、字符串、数字等)
\item \textbf{目标识别:}明确优化目标(最大价值、方案数、是否可行等)
\end{itemize}

\textbf{2. 背包类型判断}
\begin{align}
\text{背包类型} = \begin{cases}
\text{01背包} & \text{每个物品最多选择1次} \\
\text{完全背包} & \text{每个物品可以选择无限次} \\
\text{多重背包} & \text{每个物品有数量限制} \\
\text{分组背包} & \text{物品分组,每组最多选1个}
\end{cases}
\end{align}

\textbf{3. 状态设计原则}
\begin{itemize}
\item \textbf{维度选择:}通常包含"前i个物品"和"容量j"两个维度
\item \textbf{状态含义:}明确$dp[i][j]$表示的具体含义
\item \textbf{边界处理:}正确设置初始状态$dp[0][*]$和$dp[*][0]$
\end{itemize}

\textbf{4. 转移方程构建}
核心思想:对于每个物品,考虑"选择"与"不选择"的最优决策
\begin{align}
dp[i][j] = \text{optimize}(\text{不选择第i个物品}, \text{选择第i个物品})
\end{align}

\textbf{5. 循环顺序的重要性}
\begin{itemize}
\item \textbf{01背包:}容量维度逆序遍历,避免重复选择
\item \textbf{完全背包:}容量维度正序遍历,允许重复选择
\item \textbf{组合与排列:}外层循环决定是否考虑顺序
\end{itemize}

\textbf{6. 优化策略}
\begin{itemize}
\item \textbf{空间优化:}二维数组压缩为一维数组
\item \textbf{时间优化:}提前剪枝、单调队列优化等
\item \textbf{实现优化:}使用HashSet提高查找效率
\end{itemize}

\textbf{7. 解决步骤模板}
\begin{verbatim}
Step 1: 分析问题,确定背包类型
Step 2: 定义状态dp[i][j]的含义
Step 3: 找出状态转移方程
Step 4: 确定初始化条件
Step 5: 确定循环顺序
Step 6: 考虑空间优化
Step 7: 编码实现并测试
\end{verbatim}

\textbf{8. 常见变种的识别技巧}
\begin{itemize}
\item \textbf{求方案数:}状态转移用加法而非取最值
\item \textbf{求可行性:}状态用布尔值,转移用逻辑或
\item \textbf{多维约束:}增加状态维度,如$dp[i][j][k]$
\item \textbf{字符串问题:}将字符串长度视为"容量",单词视为"物品"
\end{itemize}

这种系统性的思路帮助我们在遇到新的背包变种时,能够快速识别问题类型并选择合适的解决方案。

\subsection{总结}
背包问题是动态规划的经典应用,通过状态定义、状态转移方程的建立,我们可以系统地解决这类优化问题。从基础的01背包到各种变种问题,都体现了动态规划"大事化小,小事化了"的核心思想。掌握背包问题的解法,对于理解更复杂的动态规划问题具有重要意义。

时间复杂度:$O(nW)$

空间复杂度:$O(W)$(优化后)

这些复杂度在实际应用中是可接受的,使得背包问题的动态规划解法在工程实践中具有很好的实用性。



